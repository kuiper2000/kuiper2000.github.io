\documentclass{article}
\usepackage{cv}
\usepackage{blindtext}

\begin{document}

\definecolor{airforceblue}{rgb}{0.01, 0.43, 0.72}
% CV Header section
{\huge\color{airforceblue}\textbf{Kai-Chih Tseng (Kai)}\par}
\rule{\textwidth}{0.5mm}\par

\vspace{5ex}

% Personal Details section
\begin{tabular}{
		>{\bfseries}p{.20\linewidth}
		p{.4\linewidth}
		>{\bfseries}p{.1\linewidth}
		p{.25\linewidth}
	}
	Address       & 201 Forrestal Rd,          &  Phone  & +1 (719) 201 3204 \\
	              & Princeton, NJ 08540          &    &  \\
	Email         & \href{mailto:Kai-chi.Tseng@noaa.gov}{Kai-chi.Tseng@noaa.gov} & website & \href{https://kuiper2000.github.io/}{https://kuiper2000.github.io/}    \\
                  & \href{mailto:kaichiht@princeton.edu}{kaichiht@princeton.edu} &       &     \\
	%Nationality   & Taiwan\\
\end{tabular}

\section{\color{airforceblue}Appointment}
\begin{tabular}{>{\bfseries}p{4cm}p{\linewidth-2.5cm\relax}}
	2022 (expected)   & Assistant Professor- \href{https://www.princeton.edu/}{Atmospheric Science, National Taiwan University}\par
\end{tabular}


% Education section
\section{\color{airforceblue}Education and Professional Training}
\begin{tabular}{>{\bfseries}p{4cm}p{\linewidth-2.5cm\relax}}
	2019  -- 2022 Postdoc & \href{https://www.princeton.edu/}{Atmospheric and Oceanic Science, Princeton University} and \par \href{https://www.gfdl.noaa.gov/}{NOAA Geophysical Fluid Dynamics Laboratory} \par
    \\
	2016 -- 2019 Ph.D. & Atmospheric Science -
	\href{https://www.colostate.edu/}{Colorado State University} \par
	(Highest honors and degree awarded within 3yrs) \par
	\\
	2008 -- 2014 BS/MS & Atmospheric Science - \href{https://www.ntu.edu.tw/}{National Taiwan University} \par 
	(Highest honors in both B.S. and M.S.) 

\end{tabular}


\section{\color{airforceblue}Publications}
\begin{enumerate}
	 
	\item \normalsize{\bf{\underline{Tseng K.-C}}}., and co-authors 2021: The multiseasonal forecast of CONUS tornado activity and the optimal environment for severe weather (in preparation)
	\item Bushuk, M., and \normalsize{\bf{\underline{co-authors}}} 2021: Mechanisms of Regional Artic Sea Ice Predictability in Dynamical Seasonal Forecast Systems (submitted to J. Clim.)
	\item Chen Y.-L.,and \normalsize{\bf{\underline{co-authors}}} 2021: Effect of the MJO on East Asian winter rainfall as revealed by a SVD analysis (in press)
	\item Zhang, L., and \normalsize{\bf{\underline{co-authors}}} 2021: Using large ensembles to elucidate the possible roles of Southern Ocean meridional overturning circulation in the Southern Ocean 36-yr SST trend (submitted to JClim.)
	\item Jia, L., and \normalsize{\bf{\underline{co-authors}}} 2021: Skillful seasonal prediction of North American summertime hot extremes (submitted to JClim)
	\item \normalsize{\bf{\underline{Tseng K.-C}}}., and co-authors 2021: When will humanity notice its influence on atmospheric rivers? (submitted) \par
	\item Bushuk, M., and \normalsize{\bf{\underline{co-authors}}} 2021: Seasonal prediction and predictability of regional Antarctic sea ice \textit{J. Climate}, 1--68 \par
	\item Zhang, G., and \normalsize{\bf{\underline{co-authors}}} 2021: Seasonal Predictability of Baroclinic Waves (npj-Climate and Atmospheric Science, in press) \par
    \item \normalsize{\bf{\underline{Tseng K.-C}}}., and co-authors 2021: Are multiseasonal forecasts of atmospheric rivers possible? \normalsize{\bf{48}}, e2021GL094000. https://doi.org/10.1029/2021GL094000\par
	\item \normalsize{\bf{\underline{Tseng K.-C}}}., N. C. Johnson., E. D. Maloney, E. A. Barnes, and S. B. Kapnick 2021: Mapping Large-scale Climate Variability to Hydrological Extremes: An Application of the Linear Inverse Model to Subseasonal prediction \textit{J. Climate}, \normalsize{\bf{34(11)}}, 4207--4225 \par
	\item \normalsize{\bf{\underline{Tseng K.-C}}}., E. A. Barnes, and E. D. Maloney 2021: The important role of the MJO for extratropical variability in observations and the CMIP5 climate models (submitted to JGR-Atmosphere)  \par 

	2020 \par 
	\item \normalsize{\bf{\underline{Tseng K.-C}}}., E. D. Maloney and E. A. Barnes, 2020: The consistency of MJO teleconnection patterns on interannual timescales \textit{J. Climate}, \normalsize{\bf{33}}, 3471–-3486
	\item \normalsize{\bf{\underline{Tseng K.-C}}}., E. A. Barnes, and E. D. Maloney, 2020: The importance of past MJO activity in determining the future state of midlatitude circulation  \textit{J. Climate} \normalsize{\bf{33}}, 2131–-2147 \par

	
	\item \normalsize{\bf{\underline{Tseng K.-C}}}., E. D. Maloney, and E. A. Barnes, 2019: Explaining the consistency of MJO teleconnection patterns with linear Rossby wave theory, \textit{J. Climate}, \normalsize{\bf{32}}, 531--548.
	\item \normalsize{\bf{\underline{Tseng K.-C}}}., E. A. Barnes, and E. D. Maloney, 2018: Prediction of the midlatitude response to strong Madden-Julian oscillation events on S2S timescales, \textit{Geophys. Res. Lett}., \normalsize{\bf{45}}, 463--470. (NOAA Climate Program Office research highlight)\par
	\item \normalsize{\bf{\underline{Tseng K.-C}}}., C.-H. Sui., and T. Li, 2015: Moistening Processes of MJO events during DYNAMO/CINDY, \textit{J. Climate}, \normalsize{\bf{28}}, 3041--3057.
    \end{enumerate}


\section{\color{airforceblue}Honors and Awards}
\begin{tabular}{>{\bfseries}p{2.5cm}p{10cm}p{\linewidth-2.5cm\relax}}
     2021   & YuShan Scholar - Faculty Early Career Development & Ministry of Education, Taiwan  
     \\ 
     2019   & Alumni Award (Ph.D. highest honor) & Colorado State University 
     \\
     2018   & Shrake-Culler Scholarship (outstanding academic record) & Colorado State University 
     \\  
	 2016   & Program of Research and Scholarly Excellence & Colorado State University 
	 \\
     2014  & Dean’s Award (M.S. highest honor) & National Taiwan University 
     \\
     2011 -- 2015 & International Research Fellowship & National Taiwan University 
     \\
     2012        & Dean’s Award (B.S. highest honor) & National Taiwan University  
     \\
     2012        & NICAM workshop Traveling Grant  & University of Tokyo
     \\  
     2008 -- 2011 & Presidential Award (top 5\% of the class in the academic year) & National Taiwan University 
     \\

     2009 -- 2010 & Hsu Shui-Sen Fellowship   & Changhua County, Taiwan 
     
\end{tabular}





\section{\color{airforceblue}Conference Presentations}
\begin{enumerate} 
	\item \normalsize{\bf{\underline{Tseng K.-C}}}., and co-authors, 2021 : Are multiseasonal forecasts of atmospheric rivers possible? WWRP/WCRP workshop [poster]
	\item \normalsize{\bf{\underline{Tseng K.-C}}}., N. C. Johnson., E. D. Maloney, E. A. Barnes, and S. B. Kapnick, 2021: Mapping Large-scale Climate Variability to Hydrological Extremes: An Application of the Linear Inverse Model to Subseasonal predictio, WWRP/WCRP/S2S/MJO teleconnection webinar [invited talk]
	\item \normalsize{\bf{\underline{Tseng K.-C}}}., and co-authors, 2020 : Seasonal Skillful Prediction of Western North America Atmospheric Rivers, AGU [poster]
	\item \normalsize{\bf{\underline{Tseng K.-C}}}., E. A. Barnes and E. D. Maloney, 2019 : The importance of past MJO activity in determining the future state of extratropical circulations, AGU [poster]
	\item \normalsize{\bf{\underline{Tseng K.-C}}}., E. A. Barnes and E. D. Maloney, 2018 : Explaining the consistency of MJO teleconnection patterns with linear Rossby wave theory, Second International Conference on Subseasonal to Seasonal Prediction (S2S) and Second International Conference on Seasonal to Decadal Prediction (S2D) [poster]
	\item \normalsize{\bf{\underline{Tseng K.-C}}}., E. A. Barnes and E. D. Maloney, 2017 : Prediction of North Pacific Height Anomalies During Strong Madden-Julian Oscillation Events, AGU Fall Meeting [oral]
	\item \normalsize{\bf{\underline{Tseng K.-C}}}., E. A. Barnes and E. D. Maloney, 2017 : Forecasting North Pacific Height Anomalies with the MJO on S2S timescales , 30th Conference on Climate Variability and Change/24th Conference on Probability and Statistics in the Atmospheric Sciences/16th Conference on Artificial Intelligence and its Applications to the Environmental Sciences [oral] 
	\item \normalsize{\bf{\underline{Tseng K.-C}}}., and C.-H. Sui, 2016 : Moistening Process in Observed and Simulated MJOs during DYNAMO/CINDY-(cumulus properties diagnosis), 32nd Conference on Hurricanes and Tropical Meteorology [oral] 
	\item \normalsize{\bf{\underline{Tseng K.-C}}}., and C.-H. Sui, 2014:A Diagnosis of Boundary Layer Moistening Processes for Madden-Julian Oscillations During DYNAMO IOP, 31st Conference on Hurricanes and Tropical Meteorology, AMS, 6B.C [oral] 
    \end{enumerate}


\section{\color{airforceblue}Reviewer Experience-Journal}
\begin{enumerate}
	\item Journal of Climate 
	\item npj, Climate and Atmospheric Science (Nature)
	\item Geophysical Research Letter
	\item JGR-Atmosphere
	\item Climate Dynamics
	\item Advances in Atmospheric Sciences
	\item Monthly Weatehr Review
	\item GFDL internal review
\end{enumerate}
\section{\color{airforceblue}Reviewer Experience-Proposal}
\begin{enumerate}
	\item National Science Fundation
\end{enumerate}



\end{document}
