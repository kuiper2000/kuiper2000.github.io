\documentclass{article}
\usepackage{cv}
\usepackage{blindtext}

\begin{document}

\definecolor{airforceblue}{rgb}{0.01, 0.43, 0.72}
% CV Header section
{\huge\color{airforceblue}\textbf{Kai-Chih Tseng (Kai)}\par}
\rule{\textwidth}{0.5mm}\par

\vspace{5ex}

% Personal Details section
\begin{tabular}{
		>{\bfseries}p{.20\linewidth}
		p{.4\linewidth}
		>{\bfseries}p{.1\linewidth}
		p{.25\linewidth}
	}
	Address       & No. 1, Section 4, Roosevelt Rd,          &  Phone  & +1 (719) 201 3204\\
	              & Taipei, Taiwan 10617          &    &  \\
	Email         & \href{mailto:kaichiht@ntu.edu.tw}{kaichiht@ntu.edu.tw} & website & \href{https://kuiper2000.github.io/}{https://kuiper2000.github.io/}    \\
	%Nationality   & Taiwan\\
\end{tabular}

% Education section
\section{\color{airforceblue}Positions}
\begin{tabular}{>{\bfseries}p{5cm}p{\linewidth-2.5cm\relax}}
	2022 Assistant Professor   & \href{https://www.princeton.edu/}{Atmospheric Science, National Taiwan University}\par 
	\\
	2019  -- 2022 Postdoc & \href{https://www.princeton.edu/}{Atmospheric and Oceanic Science, Princeton University} and \par \href{https://www.gfdl.noaa.gov/}{NOAA Geophysical Fluid Dynamics Laboratory} \par
    \\
	2016 -- 2019 Ph.D. & Atmospheric Science -
	\href{https://www.colostate.edu/}{Colorado State University} \par
	(Highest honors and degree awarded within 3yrs) \par
	\\
	2008 -- 2014 BS/MS & Atmospheric Science - \href{https://www.ntu.edu.tw/}{National Taiwan University} \par 
	(Highest honors in both B.S. and M.S.) 

\end{tabular}


\section{\color{airforceblue}Publications} (*=mentored students)\par
\normalsize{\bf{2025}} 
\begin{enumerate} [resume] 
	\item Chang C.-Y.,*, \normalsize{\bf{\underline{K.-C. Tseng}}}, and Y.-C. Kan Moisture and Persistence of Annular Mode: The Role of Large-Scale Tropical Variability (submitted to npj, Climate and Atmospheric Science, Nature) 
	\item Hsu, S.-P.*, H.-J. Tai, C.-H. Sui, Y.-C. Kan*, and \normalsize{\bf{\underline{K.-C. Tseng}}} 2024: The Roles of Shallow Convection and Cloud-Radiative Feedback in Convectively-Coupled Kelvin Waves (submitted to Journal of Climate)
	\item Xiao, H.-M., \normalsize{\bf{\underline{K.-C. Tseng}}}, J.-Y. Yu, H.-H. Hsu, T.-H. Lee, M.-H. Lo, 2024: The Enhanced Teleconnection of Maritime Continent Deforestation on North Pacific Climate During La Niña Conditions (submitted to Journal of Climate)
	\item Loi, C. L.,* \normalsize{\bf{\underline{K.-C. Tseng}}} and C.-C. Wu., 2024: Predictability of Tropical Cyclone Track Density in S2S Reforecast  \textit{npj Clim Atmos Sci},\normalsize{\bf{8(1)}}, 24
	\item \normalsize{\bf{\underline{Tseng K.-C}}}., R. Kuo and Y.-A. Feng., 2024: Order in Chaos: Solving the Analytical Solution of Ensemble Forecast with Data-driven Liouville equation (submitted to JAMES)
	\item \normalsize{\bf{\underline{Tseng K.-C}}}., and co-authors 2024: Skillful forecasts of springtime CONUS tornado activity up to a year in advance (submitted to npj, Climate and Atmospheric Science, Nature) 

\end{enumerate}	
\normalsize{\bf{2024}} 
\begin{enumerate} 
	\item Zhang, W., B. Xiang, \normalsize{\bf{\underline{K.-C. Tseng}}} N. Johnson, L. Harris, T. Delworth, 2024: Subseasonal prediction of wintertime atmospheric rivers in the GFDL SPEAR model. (submitted to npj, Climate and Atmospheric Science, Nature)
	\item \normalsize{\bf{\underline{Tseng K.-C}}}., and Y.-H., Ho* 2024: The Subseasoanl Predictability of North Pacific Subtropical High and the 2020 Record-breaking Event \textit{npj Clim Atmos Sci},\normalsize{\bf{7(1)}}, 53
	\item Schmitt*, J., \normalsize{\bf{\underline{Tseng, K.-C}}}, M. Hughes, and N. Johnson, 2024: Illuminating snow droughts: The future of Western United States snowpack in the SPEAR large ensemble. \textit{JGR Atmosphere} \normalsize{\bf{129 (10)}}, e2023JD039754
\end{enumerate}	
\normalsize{\bf{2023}} 
\begin{enumerate}[resume] 
	
	\item Bower, C., Serafin, K., \normalsize{\bf{\underline{K.-C. Tseng}}}, Baker, J., 2023: Atmospheric River Sequences as Indicators of Hydrologic Hazard in Present and Future Climates. \textit{Earth's Future}, \normalsize{\bf{11}} , e2023EF003536 \par
	\item Jong, B.-T., and \normalsize{\bf{\underline{co-authors}}} 2023: Investigating Observed and Projected Increases in Extreme Precipitation over the Northeast United States using High-resolution Climate Model Simulations  \textit{npj Clim Atmos Sci},\normalsize{\bf{6(18)}} \par

\end{enumerate}	
\normalsize{\bf{2022}} 
\begin{enumerate}[resume]     
	\item Jia, L., and \normalsize{\bf{\underline{co-authors}}} 2022: Skillful seasonal prediction of North American summertime hot extremes (in press), \textit{J. Climate}, \normalsize{\bf{35(13)}}, 4331--4345 (Nature research highlight)
	\item \normalsize{\bf{\underline{Tseng K.-C}}}., and co-authors 2022: When will humanity notice its influence on atmospheric rivers? \textit{JGR. Atmospheres} \normalsize{\bf{127(9)}} 
	\item Bushuk, M., and \normalsize{\bf{\underline{co-authors}}} 2022: Mechanisms of Regional Artic Sea Ice Predictability in Dynamical Seasonal Forecast Systems \textit{J. Climate}, \normalsize{\bf{35(13)}}, 4207--4231
	\item Zhang, L., and \normalsize{\bf{\underline{co-authors}}} 2022: Using large ensembles to elucidate the possible roles of Southern Ocean meridional overturning circulation in the Southern Ocean 36-yr SST trend \textit{J. Climate}, \normalsize{\bf{35(5)}}, 1577--1596
\end{enumerate}
\normalsize{\bf{2021}} 
\begin{enumerate}[resume]   
	\item Chen Y.-L.,and \normalsize{\bf{\underline{co-authors}}} 2021: Effect of the MJO on East Asian winter rainfall as revealed by a SVD analysis \textit{J. Climate}, \normalsize{\bf{34(24)}}, 9729--9746 \par
	\item Bushuk, M., and \normalsize{\bf{\underline{co-authors}}} 2021: Seasonal prediction and predictability of regional Antarctic sea ice \textit{J. Climate}, \normalsize{\bf{34(15)}}, 6207--6233 \par
	\item Zhang, G., and \normalsize{\bf{\underline{co-authors}}} 2021: Seasonal Predictability of Baroclinic Waves \textit{npj Clim Atmos Sci} \normalsize{\bf{4(50)}}  \par
    \item \normalsize{\bf{\underline{Tseng K.-C}}}., and co-authors 2021: Are multiseasonal forecasts of atmospheric rivers possible? \normalsize{\bf{48}}, e2021GL094000. https://doi.org/10.1029/2021GL094000 (GFDL research highlight)\par
	\item \normalsize{\bf{\underline{Tseng K.-C}}}., N. C. Johnson., E. D. Maloney, E. A. Barnes, and S. B. Kapnick 2021: Mapping Large-scale Climate Variability to Hydrological Extremes: An Application of the Linear Inverse Model to Subseasonal prediction \textit{J. Climate}, \normalsize{\bf{34(11)}}, 4207--4225 \par
	\item \normalsize{\bf{\underline{Tseng K.-C}}}., E. A. Barnes, and E. D. Maloney 2021: The important role of the MJO for extratropical variability in observations and the CMIP5 climate models (submitted to JGR-Atmosphere)  \par 
	 \end{enumerate}
\normalsize{\bf{2020} and prior} 
\begin{enumerate}[resume]  
	\item \normalsize{\bf{\underline{Tseng K.-C}}}., E. D. Maloney and E. A. Barnes, 2020: The consistency of MJO teleconnection patterns on interannual timescales \textit{J. Climate}, \normalsize{\bf{33}}, 3471–-3486
	\item \normalsize{\bf{\underline{Tseng K.-C}}}., E. A. Barnes, and E. D. Maloney, 2020: The importance of past MJO activity in determining the future state of midlatitude circulation  \textit{J. Climate} \normalsize{\bf{33}}, 2131–-2147 \par
	\item \normalsize{\bf{\underline{Tseng K.-C}}}., E. D. Maloney, and E. A. Barnes, 2019: Explaining the consistency of MJO teleconnection patterns with linear Rossby wave theory, \textit{J. Climate}, \normalsize{\bf{32}}, 531--548.
	\item \normalsize{\bf{\underline{Tseng K.-C}}}., E. A. Barnes, and E. D. Maloney, 2018: Prediction of the midlatitude response to strong Madden-Julian oscillation events on S2S timescales, \textit{Geophys. Res. Lett}., \normalsize{\bf{45}}, 463--470. (NOAA Climate Program Office research highlight)\par
	\item \normalsize{\bf{\underline{Tseng K.-C}}}., C.-H. Sui., and T. Li, 2015: Moistening Processes of MJO events during DYNAMO/CINDY, \textit{J. Climate}, \normalsize{\bf{28}}, 3041--3057.
    \end{enumerate}


\section{\color{airforceblue}Honors and Awards}
\begin{tabular}{>{\bfseries}p{2.5cm}p{10.5cm}p{\linewidth-2.5cm\relax}}
	2023   & 2030 Cross-Generation Young Scholars Program & NSTC, Taiwan
     \\ 
     2021   & Yushan Young Scholar - Faculty Early Career Development & Ministry of Education, Taiwan  
     \\ 
     2019   & Alumni Award (Ph.D. highest honor, best Ph.D. thesis) & Colorado State University 
     \\
     2018   & Shrake-Culler Scholarship (outstanding academic record) & Colorado State University 
     \\  
	 2016   & Program of Research and Scholarly Excellence & Colorado State University 
	 \\
     2014  & Dean’s Award (M.S. highest honor, best master thesis) & National Taiwan University 
     \\
     2011 -- 2014 & International Research Fellowship & National Taiwan University 
     \\
     2012        & Dean’s Award (Valedictorian, major GPA=4.2/4.3) & National Taiwan University  
     \\
     2012        & NICAM workshop Traveling Grant (the only undergrad recipient)  & University of Tokyo
     \\  
     2008 -- 2011 & Presidential Award (top 5\% of the class in the academic year) & National Taiwan University 
     \\

     2009 -- 2010 & Hsu Shui-Sen Fellowship   & Changhua County, Taiwan 
     
\end{tabular}





\section{\color{airforceblue}Conference Presentations}
\begin{enumerate} 
	\item \normalsize{\bf{\underline{Tseng K.-C}}}, 2023 : Infinite members of ensemble weather forecast with data-driven statistical mechanics [poster]
	\item \normalsize{\bf{\underline{Tseng K.-C}}}., and co-authors, 2022 : Mapping Large-scale Climate Variability to Hydrological Extremes: An Application of the Linear Inverse Model to Subseasonal predictio. AMS annual meeting [invited talk]
	\item \normalsize{\bf{\underline{Tseng K.-C}}}., and co-authors, 2022 : When will humanity notices its impacts on atmospheric rivers? AMS annual meeting [talk]
	\item \normalsize{\bf{\underline{Tseng K.-C}}}., and co-authors, 2021 : Are multiseasonal forecasts of atmospheric rivers possible? WWRP/WCRP workshop [poster]
	\item \normalsize{\bf{\underline{Tseng K.-C}}}., N. C. Johnson., E. D. Maloney, E. A. Barnes, and S. B. Kapnick, 2021: Mapping Large-scale Climate Variability to Hydrological Extremes: An Application of the Linear Inverse Model to Subseasonal predictio, WWRP/WCRP/S2S/MJO teleconnection webinar [invited talk]
	\item \normalsize{\bf{\underline{Tseng K.-C}}}., and co-authors, 2020 : Seasonal Skillful Prediction of Western North America Atmospheric Rivers, AGU [poster]
	\item \normalsize{\bf{\underline{Tseng K.-C}}}., E. A. Barnes and E. D. Maloney, 2019 : The importance of past MJO activity in determining the future state of extratropical circulations, AGU [poster]
	\item \normalsize{\bf{\underline{Tseng K.-C}}}., E. A. Barnes and E. D. Maloney, 2018 : Explaining the consistency of MJO teleconnection patterns with linear Rossby wave theory, Second International Conference on Subseasonal to Seasonal Prediction (S2S) and Second International Conference on Seasonal to Decadal Prediction (S2D) [poster]
	\item \normalsize{\bf{\underline{Tseng K.-C}}}., E. A. Barnes and E. D. Maloney, 2017 : Prediction of North Pacific Height Anomalies During Strong Madden-Julian Oscillation Events, AGU Fall Meeting [oral]
	\item \normalsize{\bf{\underline{Tseng K.-C}}}., E. A. Barnes and E. D. Maloney, 2017 : Forecasting North Pacific Height Anomalies with the MJO on S2S timescales , 30th Conference on Climate Variability and Change/24th Conference on Probability and Statistics in the Atmospheric Sciences/16th Conference on Artificial Intelligence and its Applications to the Environmental Sciences [oral] 
	\item \normalsize{\bf{\underline{Tseng K.-C}}}., and C.-H. Sui, 2016 : Moistening Process in Observed and Simulated MJOs during DYNAMO/CINDY-(cumulus properties diagnosis), 32nd Conference on Hurricanes and Tropical Meteorology [oral] 
	\item \normalsize{\bf{\underline{Tseng K.-C}}}., and C.-H. Sui, 2014:A Diagnosis of Boundary Layer Moistening Processes for Madden-Julian Oscillations During DYNAMO IOP, 31st Conference on Hurricanes and Tropical Meteorology, AMS, 6B.C [oral] 
    \end{enumerate}


\section{\color{airforceblue}Reviewer Experience-Journal}
\begin{enumerate}
	\item Nature Climate Change (Nature)
	\item Communications Earth & Environment (Nature)
	\item Journal of Climate 
	\item npj, Climate and Atmospheric Science (Nature)
	\item Geophysical Research Letter
	\item JGR-Atmosphere
	\item Climate Dynamics
	\item Advances in Atmospheric Sciences
	\item Monthly Weatehr Review
	\item GFDL internal review
\end{enumerate}
\section{\color{airforceblue}Reviewer Experience-Proposal}
\begin{enumerate}
	\item National Science Fundation
\end{enumerate}



\end{document}
