\documentclass{article}
\usepackage{cv}
\usepackage{blindtext}

\begin{document}

\definecolor{airforceblue}{rgb}{0.01, 0.43, 0.72}
% CV Header section
{\huge\color{airforceblue}\textbf{Kai-Chih Tseng (Kai)}\par}
\rule{\textwidth}{0.5mm}\par

\vspace{5ex}

% Personal Details section
\begin{tabular}{
		>{\bfseries}p{.20\linewidth}
		p{.4\linewidth}
		>{\bfseries}p{.1\linewidth}
		p{.25\linewidth}
	}
	Address       & 201 Forrestal Rd,          &  Phone  & +1 (719) 201 3204 \\
	              & Princeton, NJ 08540          &    &  \\
	Email         & \href{mailto:Kai-chi.Tseng@noaa.gov}{Kai-chi.Tseng@noaa.gov} & website & \href{https://kuiper2000.github.io/}{https://kuiper2000.github.io/}    \\
                  & \href{mailto:kaichiht@princeton.edu}{kaichiht@princeton.edu} &       &     \\
	%Nationality   & Taiwan\\
\end{tabular}

% Education section
\section{\color{airforceblue}Education and Professional Training}
\begin{tabular}{>{\bfseries}p{2.5cm}p{\linewidth-2.5cm\relax}}
	2019  --   & Postdoc -
	\href{https://www.princeton.edu/}{Princeton University/NOAA Geophysical Fluid Dynamics Laboratory}\par
	Advisor: Dr. Nat Johnson \par
    \\
	2016  -- 2019  & Ph.D. in Atmospheric Science -
	\href{https://www.colostate.edu/}{Colorado State University, U.S.A.}\par
	Advisors: Dr. Elizabeth Barnes, Dr. Eric Maloney \par
	Graduated with Distinguished Honor and finish degree within 3 years\par
	\\
	 & (2014 Sep -- 2015 Oct : in Military Service) \par
	\\
	2008 -- 2014 & B.S. and M.S. in Atmospheric Science -
	\href{https://www.ntu.edu.tw/}{National Taiwan University,
	Taipei, Taiwan}\par
	Advisor: Dr. Chung-Hsiung Sui \par
	GPA -- B.S./M.S. 3.96/4.00 (out of 4.00 scale) \par
	Graduated with College Honors (top 5\% in BS and MS) \par
	

\end{tabular}


\section{\color{airforceblue}Publications}
\begin{enumerate}
    \item \normalsize{\bf{\underline{Tseng K.-C}}}., N. Johnson and S. Kapnik : Seasonal forecast of atmospheric river with GFDL-Seamless System for Prediction and Earth System Research (SPEAR) model (in preparation)\par
    \item \normalsize{\bf{\underline{Tseng K.-C}}}., E. A. Barnes and E. D. Maloney : Identifying the predictability source on S2S timescales with Deep Learning. (in preparation)  \par	
	\item Barnes, E. A., and co-authors(\normalsize{\bf{\underline{Tseng K.-C}}}): Physical-guided prediction of atmospheric rivers with machine learning algorithm. (in preparation)  \par
	\item Bohar Singh, Eric D. Maloney and \normalsize{\bf{\underline{K.-C. Tseng}}}., : The influence of Tropospheric QBO on Madden-Julian Oscillations Teleconnections. (in preparation)  \par

	\item \normalsize{\bf{\underline{Tseng K.-C}}}., N. Johnson., E. A. Barnes, and E. D. Maloney, : Mapping red-noise climate variability to hydrological extreme: an application of linear inverse model. (to be submitted)  
	\item \normalsize{\bf{\underline{Tseng K.-C}}}., E. A. Barnes, and E. D. Maloney, : The important role of the MJO for extratropical variability in observations and the CMIP5 climate models (submitted)  
	\item \normalsize{\bf{\underline{Tseng K.-C}}}., E. A. Barnes, and E. D. Maloney, : The importance of past MJO activity in determining the future state of midlatitude circulation  \textit{J. Climate} 
	\item \normalsize{\bf{\underline{Tseng K.-C}}}., E. D. Maloney and E. A. Barnes, : The consistency of MJO teleconnection patterns on interannual timescales (minor revision) 
	\item \normalsize{\bf{\underline{Tseng K.-C}}}., E. D. Maloney, and E. A. Barnes, 2018: Explaining the consistency of MJO teleconnection patterns with linear Rossby wave theory, \textit{J. Climate}, \normalsize{\bf{32}}, 531--548.
	\item \normalsize{\bf{\underline{Tseng K.-C}}}., E. A. Barnes, and E. D. Maloney, 2018: Prediction of the midlatitude response to strong Madden-Julian oscillation events on S2S timescales, \textit{Geophys. Res. Lett}., \normalsize{\bf{45}}, 463--470. \par
	\item \normalsize{\bf{\underline{Tseng K.-C}}}., C.-H. Sui., and T. Li, 2015: Moistening Processes of MJO events during DYNAMO/CINDY, \textit{J. Climate}, \normalsize{\bf{28}}, 3041--3057.
    \end{enumerate}


\section{\color{airforceblue}Honors and Awards}
\begin{tabular}{>{\bfseries}p{2.5cm}p{10cm}p{\linewidth-2.5cm\relax}}
     2019   & Alumni Award (distinguished honor with best Ph.D. paper) & Colorado State University 
     \\
     2018   & Shrake-Culler Scholarship (outstanding academic record) & Colorado State University 
     \\  
	 2016   & Program of Research and Scholarly Excellence & Colorado State University 
	 \\
     2014  & Dean’s Award, (distinguished honor with best M.S. thesis) & National Taiwan University 
     \\
     2011 -- 2015 & International Research Fellowship & National Taiwan University 
     \\
     2012        & Dean’s Award, top 5\% of undergraduate students & National Taiwan University  
     \\
     2012        & NICAM workshop Traveling Grant  & University of Tokyo
     \\
                  & (the only undergraduate recipient)                                            &  \\  
     2009 -- 2011 & Presidential Award, top 5\% of the semester & National Taiwan University 
     \\

     2009 -- 2010 & Hsu Shui-Sen Fellowship (GPA=4.0)  & Changhua County, Taiwan 
     
\end{tabular}





\section{\color{airforceblue}Conference Presentations}
\begin{enumerate} 
	\item \normalsize{\bf{\underline{Tseng K.-C}}}., E. A. Barnes and E. D. Maloney, 2019 : The importance of past MJO activity in determining the future state of extratropical circulations, AGU [poster]
	\item \normalsize{\bf{\underline{Tseng K.-C}}}., E. A. Barnes and E. D. Maloney, 2018 : Explaining the consistency of MJO teleconnection patterns with linear Rossby wave theory, Second International Conference on Subseasonal to Seasonal Prediction (S2S) and Second International Conference on Seasonal to Decadal Prediction (S2D) [poster]
	\item \normalsize{\bf{\underline{Tseng K.-C}}}., E. A. Barnes and E. D. Maloney, 2017 : Prediction of North Pacific Height Anomalies During Strong Madden-Julian Oscillation Events, AGU Fall Meeting [oral]
	\item \normalsize{\bf{\underline{Tseng K.-C}}}., E. A. Barnes and E. D. Maloney, 2017 : Forecasting North Pacific Height Anomalies with the MJO on S2S timescales , 30th Conference on Climate Variability and Change/24th Conference on Probability and Statistics in the Atmospheric Sciences/16th Conference on Artificial Intelligence and its Applications to the Environmental Sciences [oral] 
	\item \normalsize{\bf{\underline{Tseng K.-C}}}., and C.-H. Sui, 2016 : Moistening Process in Observed and Simulated MJOs during DYNAMO/CINDY-(cumulus properties diagnosis), 32nd Conference on Hurricanes and Tropical Meteorology [oral] 
	\item \normalsize{\bf{\underline{Tseng K.-C}}}., and C.-H. Sui, 2014:A Diagnosis of Boundary Layer Moistening Processes for Madden-Julian Oscillations During DYNAMO IOP, 31st Conference on Hurricanes and Tropical Meteorology, AMS, 6B.C [oral] 
    \end{enumerate}


\section{\color{airforceblue}Reviewer Experience}
\begin{enumerate}
	\item Journal of Climate 
	\item npj, Climate and Atmospheric Science (Nature)
	\item Geophysical Research Letter
	\item JGR-Atmosphere
	\item Climate Dynamics
	\item Advances in Atmospheric Sciences
	\item Monthly Weatehr Review
\end{enumerate}





\end{document}
